%%%%%%%%%%%%%%%%%%%%
% Copyright and authorship: Fernando Oleo Blanco.
%
% Thanks to all the people that helped me get to where I am.
% Free use, just acknowledge the authors.
%%%%%%%%%%%%%%%%%%%%

\documentclass[12pt, a4paper]{thesis} % General document definition. In order to have the text completely centered, do not use the twoside option; however, twoside is recommended for printing.

% For general information and help, refer to https://www.overleaf.com/learn

%% PACKAGES

\usepackage[utf8]{inputenc} % Set the utf8 input encoding
\usepackage[english]{babel} % Select your prefered language. Example: \usepackage[activeacute,spanish]{babel}
\usepackage{amsmath, amsfonts, amssymb} % Mathematical notation
\usepackage{fancyhdr} % Header and footer customization
\usepackage{titlesec} % Section printing customization
\usepackage{
} % Graphics import and tools
\usepackage{geometry} % To modify the geometry of pages (margins and other lengths)
\usepackage{booktabs} % To create beautiful tables
\usepackage{hyperref} % "Advance" and easier referencing
\usepackage{listings} % To add and format code. See 
\usepackage{xcolor} % Adds more colors to the available list
\usepackage[final]{pdfpages} % Includes pdfs directly, not as images
\usepackage{lipsum} % Dummy text to test the design

\usepackage[backend=biber,
style=numeric,
citestyle=numeric,
sorting=none]{biblatex}
% Citation configuration. We use biblatex, which is more
% complex, but as customizable as it gets.
% Please, modify this to your liking.
% IMPORTANT: this requires biber to be installed and run every time
% you change the .bib file! To make it easier, if you are using TeXStudio, do the following:
% 1. Go to options. 2. Select the Build menu. 3 In standard bibliography, change bibtex to biber. 4. Profit
% These changes will allow the editor to detect any changes and "recompile" the files if needed.
% For other editors see: https://tex.stackexchange.com/questions/154751/biblatex-with-biber-configuring-my-editor-to-avoid-undefined-citations
% See link for a quick introduction: https://tex.stackexchange.com/questions/26516/how-to-use-biber/34136
% If you are having issues with the bibliography, please, search for how to install and run biber!

% Specialised packages

\usepackage{algorithm2e} % To write beautiful algorithms
\usepackage{pgfplots} % Create native LaTeX looking plots (other backends are available)
\usepackage{todonotes} % Graphically create TODO entries
\usepackage{siunitx} % Write Units in accordance with the SI organization
\usepackage{eurosym} % Official Euro symbol



%% MY PACKAGES
\usepackage{colortbl}
\usepackage{capt-of}
\usepackage{graphicx}
\usepackage{float}

%% PACKAGE CONFIGURATION

% Bibliography resource
\addbibresource{main.bib}

% Header and footer customization
\fancyhf{}

\fancyhead[R]{\slshape \nouppercase \rightmark}
\fancyfoot[C]{\thepage}
% IMPORTANT! If your chapter/section names are too long to nicely fit on the header, use the shortened variant:
% \chapter[short tile]{actual long title} or \section[short title]{actual long title, title}

%% CUSTOM COMMANDS
\setlength{\parskip}{4mm} % Espaciado entre párrafos
\lstset{breaklines}  % Para que quepa todo el códigoen el ancho de una página

% lISTINGS CODE STYLE
\usepackage{xcolor}

\definecolor{codegreen}{rgb}{0,0.6,0}
\definecolor{codegray}{rgb}{0.5,0.5,0.5}
\definecolor{backcolour}{rgb}{0.99,0.99,0.99}

\lstdefinestyle{mystyle}{
	backgroundcolor=\color{backcolour},   
	commentstyle=\color{gray},
	keywordstyle=\color{blue},
	numberstyle=\tiny\color{codegray},
	stringstyle=\color{codegreen},
	basicstyle=\ttfamily\footnotesize,
	breakatwhitespace=false,         
	breaklines=true,                 
	captionpos=b,                    
	keepspaces=true,                 
	numbers=left,                    
	numbersep=4pt,                  
	showspaces=false,                
	showstringspaces=false,
	showtabs=false,                  
	tabsize=2
}

\lstset{style=mystyle}

%% Color Links
\usepackage{xcolor}
\hypersetup{
	colorlinks,
	linkcolor={red!50!black},
	citecolor={blue!50!black},
	urlcolor={blue!80!black}
}


%% DOCUMENT
\begin{document}
	
	\frontmatter % First few pages
	\pagestyle{empty} % We suppress the "fancy" hearders in the front matter
	
	% This creates a basic title page based on the one proposed by Comillas University.
	\begin{titlepage}
		\begin{figure}
			\centering
			\includegraphics[width=0.6\linewidth]{LogoUniversidadBN}
		\end{figure}
		\centering
		\Large Máster en ingeniería de telecomunicaciones \\ % Modify accordingly
		\vspace*{2.5em} % This just adds some vertical spacing
		\centering
		Proyecto de fin de máster \\ % Modify accordingly
		\vspace*{1em}
		\textbf{Identificación y recogida de objetos con un brazo robótico utilizando técnicas de reinforcement learning} % Write here the title of your study
		\\ \large
		\vspace*{3em}
		Autor \\ \textbf{Pablo Iglesia Fernández-Tresguerres} \\ % Your name
		\vspace*{1em}
		Dirigido por \\ Philippe Juhel % Supervisor(s) name(s) 
		% {\vfill \vspace*{3em}{Author's signature:\hrulefill \hfill} \hfill \\ \vspace*{4em}Supervisors' signatures:\hrulefill \hfill} % Uncoment this line if you would like to add a signature space
		\vfill
		Madrid \\
		Enero 2021 % Modify accordingly
	\end{titlepage}
	
	% Write your abstract
	\begin{abstract}
		Abstract content
	\end{abstract}

	\newpage

	% Coment this block if you don't want a general information page
	% Thanks and other information page
	{\centering Thank yous\\}
	\vfill % Fill the page with whitespace (this is just for aesthetical reasons)
	And other important information

	\cleardoublepage % New page starts on the right hand side
	
	\tableofcontents % Create the index
	\listoffigures % Create index of figures
	\listoftables % Create index of tables
	% \lstlistoflistings % Create index for code sections
	
	\cleardoublepage % Open on right hand page (odd numbered)
	
	\mainmatter 
	
	% Activate customized headers
	\pagestyle{fancy}
	
	\chapter{Introduction}

Robotics and real life are worlds destined to meet. Today everyone has seen robots trying to behave like human beings. Many of them even look similar to a person and try to imitate the way we walk, talk or, ultimately, interact with the environment around us. 

Robots, Artificial Intelligence or other concepts such as Machine Learning have crept into our lives in just a few years. In fact, until recently, only a few visionaries like Marvin Minsky or Isaac Asimov used to speak of these concepts, and it was as part of science fiction novels. Nowadays, series like Black Mirror bring this technologies closer to the general public and make us reflect on how the future could be.

However, robots, and artificial intelligence in general, are still far from the vision that is told in the novels. They are not capable of understanding the environment around them, of learning or generalizing as we humans do. Companies and researchers are working on getting better generalization of the algorithms, but the truth is that, so far, Artificial Intelligence is only able to perform specific tasks for which they are programmed.

This project is one of those cases. The goal is to control a UR3 arm robot using Artificial Intelligence in order to pick disordered objects from a box and place them in a point of delivery. This task seems trivial, because we are used to see machines performing pick and place actions in industrial processes, but in fact, these kind of processes are normally just repeating the same action or the same rule over and over again. They are able to perform this tasks because they know apriori where these objects are or how they are placed, but they are not capable of generalizing the workflow. 

For instance, in Universal Robot free e-Learning course \cite{noauthor_formacion_nodate}, they expose the following example of an industry pick and place task. In
\autoref{fig:urconveyor} we can see how the robot is placing an object in a box located in a conveyor belt. The robot is using an infrared sensor to know that a box has arrived, and this box will always be in the same place because there is a stopper in the conveyor belt which doesn't allow the box to keep moving. On the other hand, the object is picked from the other conveyor belt using the same system to detect the arrival of a new object. The whole task is using a complex architecture, but the robot is performing the same chain of movements in a loop and the only intelligence that the robot has to have is waiting for the object and the box to come.

\begin{figure}
	\centering
	\includegraphics[width=0.85\linewidth]{Images/UR_conveyor}
	\caption[Pick and Place Task]{Universal Robots Pick and Place Task}
	\label{fig:urconveyor}
\end{figure}

To achieve generalization in this project, Reinforcement Learning (RL) together with Image Recognition techniques have been used. This algorithms give the robot the ability of calculating, for each time step, the optimal action to achieve the final goal of picking all the objects from the box and placing them in the objective point. To compute this action the robot needs to gather information about the environment such as its relative position over the box or how the pieces are distributed. This information together is called state, and the robot computes each action depending on it.

To perform this project, a distributed architecture with multiple nodes has been created. Each of them takes care of a different activity. For example, some nodes are used to control the robot, others to gather information about the current state, and others are used to train the Artificial Intelligence algorithm. This architecture has been created using ROS (Robot Operative System) and contributes to the project adding all the advantages of a microservices oriented architecture.

\section{Project Motivation}

The fourth industrial revolution is here, and it will change the way that goods are produced, raising efficiency by increasing the amount of automated processes. 
This will lead to a faster production and a reduction of errors, as machines have the ability to decide and act in fractions of seconds without making mistakes. Furthermore, machines can also be working 24 hours per day stopping just for maintenance checks, which would help to increase the productivity factor without increasing the expense in human resources.

We have been hearing about industry 4.0 since 2011, but the truth is that it is not a reality yet. We are just in the beginning, and it will take decades to perform such a big change in the industry. There are some factors to take in mind in order to analyse the evolution of the industry in the following years. The improvement on the telecommunications with the arrival of 5G networks, the moral dilemma of substituting workers for machines and the impact that this could have in the society or the improvement and implementation of AI technologies are just some of these factors. 

We have seen a lot of Artificial Intelligence algorithms applied to the industry, but the truth is that these technologies are not fully developed yet and just big companies can afford to use them in their supply chain. Besides, there are some task that are now performed by humans and cannot be done by machines due to its complexity or its importance in the whole production chain.

The motivation of this project is to contribute to the industry change providing an open source solution to a complex problem such as disordered pick and place task. This open source solution does not currently exist in the industry and would add value being a good starting point for bigger projects in the future. % Introducción
	\chapter{Description of Technologies}
	Describir las tecnologías, protocolos, herramientas específicas, etc. que se vayan a tratar durante el proyecto para facilitar su lectura y comprensión.
	Hablar de Java no procede aquí porque todo el mundo sabe lo que es, pero si en el proyecto hablo continuamente del protocolo Baseband, debo especificar en este capítulo qué es y para qué sirve.

	\begin{itemize}
	\item[\textendash]ROS + catkin  % En profundidad
			The first decission we had to make was about the architecture of the project. Was it a good idea to build all the project in the same computer? How should we communicate with the Robot?

		We found the best solution for this questions in ROS (Robot Operative System), which is a framework for the development of software for robots that provides the functionality of an operating system in a heterogeneous cluster. ROS was originally developed in 2007 under the name switchyard by the Stanford Artificial Intelligence Laboratory to support the Stanford Artificial Intelligence Robot (STAIR) project. Since 2008, development has continued primarily at Willow Garage, a robotic research institute with more than twenty institutions collaborating on a federated development model.

		ROS provides the standard services of an operating system such as hardware abstraction, low-level device control, implementation of commonly used functionality, message passing between processes, and package maintenance. It is based on a graph architecture where the processing takes place in the nodes that can receive, send and multiplex messages from sensors, control, states, schedules and actuators, among others. The library is oriented for a UNIX system (Ubuntu (Linux)) although it is also adapting to other operating systems such as Fedora, Mac OS X, Arch, Gentoo, OpenSUSE, Slackware, Debian or Microsoft Windows, considered as 'experimental'.
		ROS is free software under BSD license terms. This license allows freedom for commercial and investigative use. Contributions of packages in ros-pkg are under a variety of different licenses.

		All of these features made ROS ideal for the project. Specially the following ones:
			- Universal Robots drivers for ROS using mooveit make it possible to controll ROS remotelly.
			- ROS is a multi-node oriented framework, which allows us to take all the advantages of micro-services. We can split the software by functionallity gaining:
				○ The possibility of giving more or less computation power to each functionality depending on its needs. In the project we have useed from  computers with the better Nvidia Graphic card and 32 GB of RAM to other mini-computers such as a Raspberry-pi or an Arduino Card.
				○ The chance of using a different environment for each functionality. Different versions of python and even different programming languages, diferent versions of libraries, diferent Operative Systems, etc.
				○ Isolation of each component of the project, allowing us to develop separatly each functionality without affecting the rest of the functionalities of the project.
			- It can work over an Arduino Card. It is not self sufficient, as it needs to be serial connected to a computer (Or Raspberry pi) in orther to work. We needed the arduino card in order to build our "home made" vaccuum gripper.
			- It is open source and have a huge community, so we could reuse some already developed solutions such as the usb_cam package, which let us take pictures from a camara connected to another node or computer.

		Catkin is the current ROS build tool, having replaced rosbuild. catkin is based on CMake, is cross-platform, open source and independent of the programming language.


	\item[\textendash]pytorch
	
	PyTorch is a Python package designed to perform numerical calculations using tensor programming. It also allows its execution on GPU to speed up calculations.

	Typically PyTorch is used both to replace numpy and process calculations on GPUs and for research and development in the field of machine learning, mainly focused on the development of neural networks. In this case we will use PyTorch in both the Reinforcement Learning algorithm and the image Processing.

	PyTorch is a very recent library and despite this it has a large number of manuals and tutorials where to find examples. In addition to a community that grows by leaps and bounds.
	PyTorch has a very simple interface for creating neural networks despite working directly with tensors without the need for a library at a higher level such as Keras for Theano or Tensorflow.
	Unlike other packages like Tensorflow, PyTorch works with dynamic graphs instead of static ones. This means that at runtime the functions can be modified and the calculation of the gradient will vary with them. On the other hand, in Tensorflow we must first define the computation graph and then use the session to calculate the results of the tensors, this makes it difficult to debug the code and makes its implementation more tedious.
	PyTorch has support to run on graphics cards (GPU), it uses internally CUDA, an API that connects the CPU with the GPU that has been developed by NVIDIA.

	\item[\textendash]arduino
		Arduino is an open source electronics creation platform, which is based on free hardware and software, flexible and easy to use for creators and developers. This platform allows the creation of different types of single-board microcomputers that can be used by the developer community for different types of use.

		As commented before, we will use Arduino Card in order to build a Vaccuum Gripper for the Robot. It will be connected with all the other nodes using ROS Queues.


	\item[\textendash]github
	
			GitHub, Inc. is a provider of Internet hosting for software development and version control using Git. It offers the distributed version control and source code management (SCM) functionality of Git, plus its own features. It provides access control and several collaboration features such as bug tracking, feature requests, task management, continuous integration and wikis for every project. Headquartered in California, it has been a subsidiary of Microsoft since 2018.
		GitHub offers its basic services free of charge. Its more advanced professional and enterprise services are commercial. Free GitHub accounts are commonly used to host open-source projects. As of January 2019, GitHub offers unlimited private repositories to all plans, including free accounts, but allowed only up to three collaborators per repository for free. Starting from April 15, 2020, the free plan allows unlimited collaborators, but restricts private repositories to 2,000 minutes of GitHub Actions per month. As of January 2020, GitHub reports having over 40 million users  and more than 190 million repositories  (including at least 28 million public repositories),  making it the largest host of source code in the world.

	\item[\textendash]CUDA
			CUDA stands for Compute Unified Device Architecture, which refers to a parallel computing platform including a compiler and a set of development tools created by nVidia that allow programmers to use a variation of the language. C programming for encoding algorithms on nVidia GPUs.
		Through wrappers you can use Python, Fortran and Java instead of C / C ++.
		Works on all nVidia GPUs from the G8X series onwards, including GeForce, Quadro, ION, and the Tesla line.1 nVidia claims that programs developed for the GeForce 8 series will also work without modification on all future nVidia cards, thanks to binary compatibility.
		CUDA tries to exploit the advantages of GPUs over general purpose CPUs by using the parallelism offered by its multiple cores, which allow the launch of a very high number of simultaneous threads. Therefore, if an application is designed using multiple threads that perform independent tasks (which is what GPUs do when processing graphics, their natural task), a GPU will be able to offer great performance in fields that could range from computational biology to biology. crypto, for example.
		The first SDK was released in February 2007 initially for Windows, Linux, and later in version 2.0 for Mac OS. It is currently offered for Windows XP / Vista / 7/8/102, for Linux 32/64 bits3 and for macOS4.

	\item[\textendash]moveit
	\item[\textendash]UR driver	
	\item[\textendash]UR3 robot
	\item[\textendash]anaconda
		Anaconda is a free and open distribution of the Python and R languages, used in data science, and machine learning. This includes processing of large volumes of information, predictive analytics and scientific computations. It is aimed at simplifying the deployment and management of software packages.
		The different versions of the packages are managed through the conda package management system, which makes it quite easy to install, run, and update data science and machine learning software such as Scikit-team, TensorFlow and SciPy.
		The Anaconda distribution is used by 6 million users and includes more than 250 data science packages valid for Windows, Linux, and MacOS.
	\end{itemize}
	
 % Description of Technologies
	\chapter{State of the art}
	
	The pick and place task that is intended to be performed in this thesis is really useful for a lot of applications into the industrial world because it would bring a lot of flexibility for these processes. A example of this applications could be an assembly line, where robotic arms could be picking all the different pieces to assemble in the product using always the same algorithm.
	
	Big companies are developing a lot of Artificial intelligence use cases in the industry, and they try to contribute to the AI community by publishing scientific articles on how they managed to use AI for their specific tasks. Unfortunately, although some companies have already developed their own solutions for our specific pick and place task, none of them have published a scientific article on the subject, making it difficult to study the way they have achieved it.
	
	One of the companies that has already developed a pick and place task is the Japanese automation company Fanuc, which has developed an AI-based solution together with Preferred Networks. As commented before, they have not published any scientific article about the topic but we can see the system working in a video they have posted on YouTube \cite{preferred_networks_inc_bin-picking_2015}. That means that we have to gather all the possible information from the video, where we could find that they have not used a Reinforcement Learning algorithm but just a Deep Neural Network (DNN) with image recognition. 
	
	To train the net, they have collected "success" or "fail" labelled images by making pick actions in random places of the box. Once they gathered a big enough dataset of images, they have trained a Deep Neural Network as the one we see in \autoref{fig:fanucdnn}, where we can also see that the Neural Net has been trained to predict whether the robot is going to success in a pick action in a specific place or not. Using that net, they can make a heat map of the whole box, predicting the points of maximum probabilities of succeed. As usually, they noticed that the bigger the image dataset, the higher the success ratio. In eight hours, they reach 90\% of success, which they say is bigger than the human success ratio.
	
	\begin{figure}
		\centering
		\includegraphics[width=0.85\linewidth]{Images/FANUC_DNN.png}
		\caption[Fanuc DNN]{Deep Neural Network of Fanuc solution (taken from the video)}
		\label{fig:fanucdnn}
	\end{figure}
	
	\subsection{Reinforcement Learning}
	
		The idea of the project is to keep using image recognition techniques but, in our case, applied to a \textbf{Reinforcement Learning} Algorithm which is an area of machine learning inspired by psychology behavioural. Its goal is to determine what actions a software agent should choose in a given environment in order to maximize some notion of "reward" or accumulated prize. 
		
		Explained easily, RL is used to make an \textbf{agent} (the robot) learn how to interact with a \textbf{environment} in order to perform a task. To achieve this, Markov Decision Process (MDP) which provides a mathematical framework for modeling decision making in situations where outcomes are partly random and partly under the control of a decision maker.
		
		\subsubsection{Markov Decision Process (MD)}
			In MDP, the environment is what we are actually trying to simulate with the MDP. The agent will interact with it to learn how to perform the task, so these are the attributes of the environment:
			\begin{itemize}
				\item[\textendash]\textbf{Agent:} The agent is the most important piece of the algorithm because it represents the objects that we want to become smarter.
				\item[\textendash]\textbf{Actions (A):} The agent can interact with the environment by performing a set of actions which is normally finite.
				\item[\textendash]\textbf{States (S): } Each time the agent performs an action, it moves to a new state. States are basically the set of information that differentiates the situation of the agent before and after performing an action. States can be transitional or terminal, when the agent meets the objective or when it gets to a forbidden position.
				\item[\textendash]\textbf{Rewards (R):} Each time an action is performed, the agent receives a reward. This reward can be positive, negative or null depending on the impact of the action to achieve the objective.
				\item[\textendash]\textbf{Policy ($\pi$):} The policy is used to define the optimal action for each step. It gives a punctuation for all of the actions in the current step as shown in the following formula. The agent takes the action with highest punctuation.
				\begin{gather*}
					\pi(a|s)=P_r\{A_t=a|S_t=s\}
				\end{gather*}
			
			\end{itemize}
			
			The MDP is divided in discrete timesteps (t), where each timestep does not have to last the same time as the previous step. Each timestep, the agent uses the policy $\pi$ to decide the next action. 
			
			Once the action is taken:
			\begin{itemize}
				\item[\textendash]The environment transits to th next state: \textbf{ \boldmath$S_t = S_{t+1}$}.
				\item[\textendash] Environment produces a new reward, which can be represented with the following formula: \boldmath
				\begin{gather*}
					P(s', r, s, a) = P_{r}\{S_{t+1}=s', R_{t+1}=r, S_{t}=s, A_{t}=a\}
				\end{gather*}
			\end{itemize}
			
			Agent's performance is calculated in terms of its future accumulated rewards known as return. This is called \textbf{expected return} an is calculated as  shown in the formula below, where $\gamma$ is the discount factor, and is used to give a bigger value to the closest steps.
			
			\begin{gather*}
				G_t = \sum_{k=t} \gamma ^{k-t} \cdot R_{k+1} \:\: \forall \:\: \gamma \in [0, 1]
			\end{gather*}
		
		
		\subsubsection{Q-Learning}
			
			Now that we know all these concepts, we have to learn what Reinforcement Learning Algorithms do to learn. Basically, \textbf{the goal of the agent is to find a policy that maximizes the expected return }. This can be done using different strategies as:
			\begin{itemize}
				\item[\textendash]\textbf{Q-Learning:} Estimating action values using Q Tables or other methods
				\item[\textendash]\textbf{TRPO:} Parametrizing the policy and optimizing its parameters
			\end{itemize}
			
			Basic Q-Learning is based on the assumption that both actions and states are limited and that the same action in the same state always drives to the same new state. Having this in mind, Q-learning algorithms build two matrices of shape \textbf{length(actions) x length(states)} as shown in the \autoref{fig:q-matrix}.
			
			\begin{figure}[h]
				\centering
				\includegraphics[width=0.7\linewidth]{Images/Q-Matrix}
				\caption[Q-Matrix]{Reward and Q Matrix shape in Basic Reinforcement Learning}
				\label{fig:q-matrix}
			\end{figure}
			
			In these two matrices, Q-Learning algorithm stores in the R matrix the reward for the pair of action-state while in the Q matrix they store cumulated reward for this same pair. The Q matrix is the one used to decide which action to perform in each state and R matrix the one used to calculate the reward of each action. 
			
			However, for the aim of this project, the states of the agent can be different in each timestep. The state would actually be partially formed by images, so the number of states can be infinite. We need a more complex version of Reinforcement Learning.
			
	\subsection{Deep Reinforcement Learning}
		
		The approach of mixing both image recognition and RL is called Deep Q Learning (DQN) or Double Deep Q Learning (DDQN) depending on the implementation and uses Neural Networks in two different stages of the algorithm. Firstly, a Convolutional Neural Network (CNN) is used to extract image features, and then, a Deep Neural Network (DNN) is used to calculate the q value of each independent action and select the next one using these values.
		
		DQL was proposed in 2012, and, since then, it has been used for a lot of different purposes. For example, Guillaume Lample and Devendra Singh Chaplot demonstrated back in 2017 that a RL agent could play FPS Games using as inputs just game scores and pixels from the screen \cite{lample_playing_2018}. Another really interesting example is this robot \cite{zhu_target-driven_2017}, which is capable of moving around a house looking for an objective and avoiding obstacles using DDQL.
		
		A good resource to understand how Reinforcement Learning really works is Deeplizard's tutorial \cite{noauthor_reinforcement_nodate}. In this tutorial they explain different versions of the algorithm and how to implement them in python to solve different OpenAI gym environments \cite{openai_gym_nodate}. 
		
		Deep Reinforcement Learning is though an union between RL and image recognition, but let's see how it actually works. The main idea is to replace the Q-table that we saw before for a Dense Neural Network that uses as input another Neural Network, a Convolutional Neural Network (CNN). The full algorithm would have as many outputs as allowed actions. Therefore, simplifying, these outputs re equivalent to the q-values saw before and so we will call them. To see it graphically, when the agent wanted to take an action, he  would pass the state image through the Neural network represented in \autoref{fig:deepqnetwork} and would take the action with higher q-value.
		
		\begin{figure}[h]
			\centering
			\includegraphics[width=0.7\linewidth]{Images/DeepQ-Network.jpg}
			\caption[Deep Q Learning]{Deep Q Learning Representation with 4 outputs}
			\label{fig:deepqnetwork}
		\end{figure}
		
		When I said "simplifying" in the previous paragraph, I meant "simplifying a lot" in the next paragraphs I will explain all the intermediate steps in the algorithm and why they are important:
		
		\begin{itemize}
			\item[\textendash]Episodes and Steps
			\item[\textendash]Exploration vs Exploitation trade-off
			\item[\textendash]Replay Memory
			\item[\textendash]Bellman's Equation
			\item[\textendash]Target and Policy Networks
		\end{itemize}
	
		\subsubsection{Episodes and Steps}
			RL training is divided in Episodes. One Episode is the sequence of actions needed to reach a terminal State. Each time the agent reaches a terminal state, an episode is ended, and a new one is started.
			
			On the other hand, steps represents every time that a new action is taken, so the number of steps taken by the agent during training is infinite. Later on, we will use as metric of performance the number of steps per episode, as they must decrease during the training.
		
		\subsubsection{Exploration vs Exploitation trade-off}
			In Reinforcement Learning there are two important concepts that are \textbf{Explore} and \textbf{Exploit}. To explore is basically gather new information about the environment and to exploit is to make the best decision with the information that we already have.
			
			In Deep Reinforcement Learning, the agent exploit the information gathered by using the pre-trained Neural Network to decide next action. On the other hand, the agent explore the environment by deciding next action randomly. We use exploration mainly in the beginning of the training because we want the agent to gather as much information of the environment as possible before starting training.
			
			When the agent uses exploitation, it is also gathering information about the environment. However, we could not let the agent explore this way because during the exploration phase we want all the actions to be performed with the same probability and neural network bias can cause some actions to be performed much more than others.
			
			So, how do we decide when the agent must explore or exploit? To decide it we can use multiple techniques, but the most common one is the Epsilon-Greedy Strategy. This strategy basically consist on setting a probability of exploring and keep decreasing it slowly during the training. It works this way:
			
			\textbf{\begin{enumerate}
				\item We set the initial exploring probability ($\epsilon$)
				\item We set the per-step epsilon decay, ($\epsilon\_decay$)
				\item For each step:
				\begin{enumerate}
					\item With probability $p = \epsilon$, the agent explores the environment (takes a random action). If not, it exploit the information by deciding the action using the NN.
					\item Whether the agent has explore or not, we decrease the probability of exploring the environment in the next step ($\epsilon = \epsilon - \epsilon\_decay$)
				\end{enumerate}
			\end{enumerate}}
			
			
		
		\subsubsection{Replay Memory}
		
			Everytime that the agent performs an action, either by exploring or exploiting, the agent lives an experience. For the purpose of training the algorithm, we will store all these experiences.
			
			Experiences are formed by the initial state, the action taken, the state reached (final state) and the reward gotten and they are stored in the Replay Memory. Then, every time that an action is taken, the algorithm is trained following this steps:
			
			\begin{enumerate}
				\item Replay Memory checks if the number of experiences is higher than the batch size
				\item If there are enough experiences:
				\begin{enumerate}
					\item Replay Memory supplies a random set of experiences of size=batch\_size.
					\item With this set of experiences, the target network is trained.
				\end{enumerate}
			\end{enumerate}
			
			Optimizing Replay Memory can be a challenge, because, if we are using a Graphic Card in the training, we would be storing all the experiences in its memory. But, why do we need to store all the experiences? We could also be using the last N experiences to train the network and it would be a less memory-consumption demanding solution. The answer to this question is tha
		\subsubsection{Bellman's Equation}
		
		
		\subsubsection{Target and Policy Networks}
		
		
		
		
		
		
		
	\subsection{Problems of Deep Reinforcement Learning in Real-world}
		
		Real-world problems introduces some challenges that we will have to manage. In march 2018, A. Rupam Mahmood, Dmytro Korenkevych, Brent J. Komer, and James Bergstra explained the problems they found while implementing a RL algorithm in a UR5 robotic arm \cite{mahmood_setting_2018}.
		
		Some of the problems they found were the following:
		
		\begin{itemize}
			\item[\textendash]Slow rate of data-collection, as movements in the real robot are slower than in a simulated environment. 
			\item[\textendash]Partial observability. Sensors cannot retrieve all the information about the environment.
			\item[\textendash]Noisy sensors will provide inaccurate information.
			\item[\textendash]Safety of the robot and its surroundings have to be taken in mind.			
			\item[\textendash]Fragility of robot components.
			\item[\textendash]Delay between an action is requested and the time it is actually performed can affect the training.
			\item[\textendash]Preparing the robot is a really difficult task:			
			\begin{itemize}
				\item[\textendash] Controlling the robot.
				\item[\textendash] Define all aspects of the environment.
				\item[\textendash] Difficulties for obtaining random and independent state when episode ends.
			\end{itemize}
		\end{itemize}
		
		Another problem that can be found in our project is that, as objects are randomly placed, the environment that the agent will have to face will be completely different each time. In fact, the robot can interact with the environment, as it can move the pieces trying to pick them, so we are facing a dynamic environment RL problem. A good example of a dynamic environment problem is the path planning of a self driven car, where each time the agent takes an action the environment will change and, furthermore, obstacles do not have to be static, but they can also move.
		
		%% TODO: Explicar en profundidad (En descripción de las tecnologías)
		There are multiple examples of articles on this topic, such as the one Xiaoyun Lei, Zhian Zhang, and Peifang Dong published in September 2018 using a DDQN approach to solve it \cite{lei_dynamic_2018}. However, there are other solutions as the one proposed by Marco A. Wiering \cite{wiering_reinforcement_2001}, where he introduces some prior knowledge to the model in order to facilitate the learning. His algorithm had problems generalizing the environment, so he introduced some prior information about the model together with a Model-based RL. This made the algorithm more capable to learn without loosing a lot of trainable capability.
		
		%% The task they performed is really different from the one we are implementing, but, as they are using the same robot, most of the problems they found are valid for our project. The solution for these problems will be explained later.
		
		%% TODO: Comentar más artículos de investigación
 % State of the art
	\chapter{Definition of the Project}

	\section{Motivation}
		The project motivation is the natural continuation of a previous project performed at ICAM University. This project was part of the assembly line of a car manufacturing process and its objective was to pick some specific plastic pieces and place them into the product. To achieve this, the system used opencv image processing, so it was recognising a specific shape given apriori.
		
		This project was totally functional and the robot could perform the task with a high successful rate. However, the lack of generalization of the system makes it hard to introduce changes as using it for another part of the assembly line. Each time that this happened someone would have to introduce the shape of the pieces to the system and to calibrate the camera to the new environment. 
		
		The motivation of the project is to create from scratch a new solution for performing the picking of the pieces. This time, the project will not be sponsored by any company, so there will be less resources to use.
		
		With this new approach, the idea is to use all the knowledge of previous documented projects on Artificial Intelligence in the industry and make a little contribution to the huge advance of industry 4.0. In fact, the idea is to make the project completely replicable so that anyone could use this project as a starting point for new applications.

	\section{Objectives}
		\todo{Revisar objetivos y adaptarlos al proyecto}
		The objectives of the project are five and are listed and explained below:
		
		\begin{itemize}
			\item[\textendash]Implement a bin picking simple solution. A basic one, without Artificial Intelligence.
			\item[\textendash]Improve the performance using RL and Image Recognition.
			\item[\textendash]Study the usage of new technologies to add information to the system. Adding physics or new image recognition techniques.
			\item[\textendash]Test different tools such as Multiflash or 3D images to improve performance.
			\item[\textendash]Add value to the scientific community making the solution available if the previous objectives are reached.
		\end{itemize}
	
	\section{Methodology}
		This project will be performed using an agile methodology, which is one of the simplest and effective processes to turn a vision for a business need into software solutions. Agile is a term used to describe software development approaches that employ continual planning, learning, improvement, team collaboration, evolutionary development, and early delivery. It encourages flexible responses to change \cite{noauthor_agile_nodate}.
		
		In the case of this project, the team is just formed by two workers and a project manager. This make necessary to make some changes to the typical agile methodology. For example, daily meetings are substituted by constant communication between both workers and weekly meetings with the project director. With this approach, all the members of the project are updated about how it is going and have clear objectives.
		
		Likewise, the methodology of this project is based in three fundamental principles as it is shown in \autoref{fig:methodology}. The first two principles are highly related, as the project is iterative because it is experimental. That means that the way of working is perform little sprints with new functionalities, test them (experimental) and, depending on the results, define the new sprint, execute it and test again (iterative). Besides, the project is also incremental because the idea is starting implementing the simplest possible solution and keep adding new improvements to it in a iterative loop until the optimal configuration is reached.
		
		\begin{figure}
			\centering
			\includegraphics[width=0.85\linewidth]{Images/methodology}
			\caption[Methodology]{Methodology}
			\label{fig:methodology}
		\end{figure}
		
	
	\section{Planning and budget}
	
			Regarding the planning, there are some really important functionalities that have been defined since the beginning of the project, as they are needed. These functionalities are split in three different groups: Hardware implementation, Artificial Intelligence Implementation and Robot controller implementation. The tasks related to these three groups are shown in \autoref{fig:planning1}, \autoref{fig:planning2} and \autoref{fig:planning3}. 
		
		\begin{figure}
			\centering
			\includegraphics[width=0.85\linewidth]{Images/planning1}
			\caption[Chronograph HW]{Chronograph of the Hardward implementation}
			\label{fig:planning1}
		\end{figure}
		
		
		\begin{figure}
			\centering
			\includegraphics[width=0.85\linewidth]{Images/planning2}
			\caption[Planning AI]{Planning of the Artificial Intelligence implementation}
			\label{fig:planning2}
		\end{figure}
		
		
		\begin{figure}
			\centering
			\includegraphics[width=0.85\linewidth]{Images/planning3}
			\caption[Planing Robot Controller]{Planning of the Robot Controller implementation}
			\label{fig:planning3}
		\end{figure}
		
		\todo{Añadir presupuesto} % Definition of the Project
	\chapter{Developed System}
	En este capítulo es donde el alumno debe describir su proyecto. En función del tipo de proyecto la estructura interna variará. El título del mismo, así como sus apartados, son sólo una sugerencia que cada alumno deberá adaptar particularmente a su proyecto.
	
	A su vez, este capítulo podrá extenderse en varios capítulos más, por ejemplo si mi proyecto consta de varias fases o módulos, lo lógico sería tener varios capítulos donde se describa el desarrollo del proyecto:

    \begin{itemize}
    	\item Capítulo 5. Implantación y configuración de la plataforma 
    	\item Capítulo 6. Desarrollo del sistema
    \end{itemize}
	
	\section{Análisis del sistema}
		...
	\section{Diseño}
		...
	\section{Implementación}
		... % Developed System
	\chapter{Results Analysis}
	Destacar los resultados más relevantes del proyecto y hacer un análisis crítico de los mismos. También es un capítulo obligatorio y clave.
 % Results Analysis
	\chapter{Conclusions and Future Work}
	Comentar las conclusiones del proyecto, destacando lo que se ha hecho, dejando claros qué objetivos se han cubierto y cuáles son las aportaciones hechas.  % Conclusions and Future Work	

	

	
	%% Include more chapters and or documents.
	
	
	\appendix
		
	\section{Robot Controller}
		En esta sección se muestra el código de algunos de los principales elementos del módulo Robot Controller, implementado en la arquitectura del proyecto.
		
		Este código también está disponible en el siguiente repositorio de github:  
		
		\href{https://github.com/pabloiglesia/robot_controller}{https://github.com/pabloiglesia/robot\_controller}
		\subsection{main.py}
			\lstinputlisting[
			language=Python, 
			% firstline=37, 
			% lastline=45
			]{D:/Proyectos/robot_controller/src/main.py}
	
		\subsection{Robot.py}
			\lstinputlisting[
			language=Python, 
			% firstline=37, 
			% lastline=45
			]{D:/Proyectos/robot_controller/src/Robot.py}
			
			
	\section{Artificial Intelligence Manager}
		En esta sección se muestra el código de algunos de los principales elementos del módulo AI Manager, implementado en la arquitectura del proyecto.
		
		Este código también está disponible en el siguiente repositorio de github: 
		
		\href{https://github.com/pabloiglesia/ai_manager}{https://github.com/pabloiglesia/ai\_manager}
			
		\subsection{main.py}
			\lstinputlisting[
			language=Python, 
			% firstline=37, 
			% lastline=45
			]{D:/Proyectos/ai_manager/src/ai_manager/main.py}
			
		\subsection{RLAlgorithm.py}
			\lstinputlisting[
			language=Python, 
			% firstline=37, 
			% lastline=45
			]{D:/Proyectos/ai_manager/src/ai_manager/RLAlgorithm.py}
	
		\subsection{Environment.py}
			\lstinputlisting[
			language=Python, 
			% firstline=37, 
			% lastline=45
			]{D:/Proyectos/ai_manager/src/ai_manager/Environment.py}
			
		\subsection{ImageController.py}
			\lstinputlisting[
			language=Python, 
			% firstline=37, 
			% lastline=45
			]{D:/Proyectos/ai_manager/src/ai_manager/ImageController.py}
			
			 % Anexos
	
	%% Include appendixes
	
	\cleardoublepage
	
	\printbibliography[heading=bibintoc] % We print the entire bibliography of the document. The author recommends that, if the document contains tremendous amounts of references, that they should be written at the end of each chapter. Refer to https://www.overleaf.com/learn/latex/Bibliography_management_in_LaTeX for more information.
	
	
\end{document}
